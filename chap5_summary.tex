\chapter{Summary, Conclusions, and Future Work \label{sec:summary}}
%observing low density super Neptunes
In this we thesis, I presented simultaneous multi-color (g-,r-, and z-bands) transit observations of low density hot Jupiters, namely HAT-P-44b, HAT-12b, and WASP-21b, using the MuSCAT instrument. %Simultaneous observation in 
The goal was to improve our understanding of these systems by refining transit parameters and searching for broad atmospheric signatures such as Rayleigh scattering in the optical wavelengths for the first time towards these systems. 

In Chapter \ref{sec:intro} a general overview of transit observation of hot Jupiters and the need for follow-up of these systems. In Chapter \ref{sec:obs} the observation and the transit modeling are discussed in detail. For the transit analysis, a Bayesian approach is implemented to model transit light curves in all bands simultaneously including wavelength-dependent systematics. This approach is useful in incorporating prior information that properly takes into account the uncertainties in the resulting best-fit transit parameters.
%results
In Chapter \ref{sec:results}, the results of transit modeling are presented which are in general agreement with previous results. % improvement in transit ephemeris. 
Careful analysis of transit depth variation in each band show a marked increase in the planetary radius from the red toward the blue ends of the visible wavelength range. For HAT-P-12b and HAT-P-44b, the measured transit depths in $g$-band have a larger Rp/Rs value than the $r$-band. Archival data using HST/STIS was used to confirm that the results of transit modeling is consistent with the accepted model spectrum for HAT-P-12b.

In Chapter \ref{sec:discussion}, spectrum model calculation for each exoplanet are presented considering two cases: (case 1) 1 $\times$ Solar and (case 2) 100 $\times$ Solar metallicity clear atmospheres, assuming thermochemical equilibrium compositions and isothermal structures. Comparing the measured transit depths with the spectrum model for HAT-P-12b and HAT-P-44b help rule out stellar activity and site-specific systematics as the source of this variation. 
To test the significance of (non-)detection of Rayleigh slope, a Monte Carlo fitting routine was performed to compute slopes directly from marginalized the transit depth distributions for each band. Results show that Rayleigh slope detection is marginal (1.5$\sigma$) for HAT-P-12b and (1.8$\sigma$) HAT-P-44b implying that the achieved photometric precision is not enough to robustly distinguish between the two atmospheric models. %Assuming that detection is true, we provide an estimate of the atmospheric scale height and the atmosphere's mean molecular weight.
However, our modeling is useful to categorically rule out the possibility that the observed trend in WASP-21b spectrum is not atmospheric in origin. Instead, the observed trend can be explained by unocculted spots on the surface of WASP-21 with a spot coverage of 0.5\%. Motivated by our results, we search for new targets feasible for future observation aimed at detecting broad atmospheric features. The new targets identified, including WASP-107, have already been considered for future observation.

%conclusion
%We conclude that the rise toward the blue end of the transmission spectrum of HAT-P-12b is due to an increase in the planetary radius, indicative of Rayleigh scattering in the planet’s atmosphere. In light of previous observations in other wavelengths, it is possible to distinguish between a H/He-dominated atmosphere covered by high-altitude clouds and hazes, which obscure absorption features in the near-IR but give rise to a steep Rayleigh scattering slope in the visible, respectively.

%Unfortunately, our achieved photometric precision is not sufficient to distinguish whether the atmosphere of HAT-P-44b and HAT-P12b can be explained by either a 1-or 100-solar metallicity abundance. 

Nevertheless, the existing data set does not constrain the atmospheric composition of HAT-P-44b beyond indicating a low-metallicity atmosphere. One way forward is through additional transit spectroscopy observations in the near- and mid-infrared, where the absorption features expected from water and carbon molecules may be detected as long as any clouds are at sufficiently low altitude. If with more precise measurements the spectrum remains featureless in this wavelength range, this would confirm the presence of haze high in the atmosphere, but would also limit the chances of further constraining this planet’s atmospheric composition.

%Our study illustrates how ground-based observations with a two-meter class telescope can provide broadband transmission spectra capable of probing atmospheric properties of exoplanets larger than super-Earths.

%The relatively quick means of planet validation and characterization using multi-color photometry has proven to be very powerful method for studying interesting transiting exoplanets during the Kepler/K2 era. This technique will be fruitful especially in the forthcoming Transiting Exoplanet Survey Satellite TESS (Ricker et al. 2015), for which MuSCAT and other ground-based transit follow-up observatories are laying the groundwork. 
 
%future work
%This study presents a continuous work to characterize transiting exoplanet systems.
The analysis employed in this work will be very useful for the near future studies of observing broad atmospheric features in exoplanet atmospheres  including Rayleigh scattering signature using a newer instrument called MuSCAT2. MuSCAT2 underwent its first light observation in late 2017 (Narita et al., in prep) and is currently conducting intensive follow-up studies of interesting transiting systems including a search for Rayleigh scattering signature. 

In addition to MuSCAT2’s improved capabilities than MuSCAT and better observing conditions in Spain, its operation is timely with the launch of the Transiting Exoplanet Survey Satellite (TESS). Unlike Kepler/K2 however, TESS will conduct all-sky survey of transiting planets around bright and nearby stars (Ricker et al. 2015). TESS is expected to discover about 300 planets smaller than 2 Earth-radii, of which 165 planets are orbiting M dwarfs and about 20 planets are orbiting the HZ (Sullivan et al. 2015). Thus,  there is a very high prospect for MuSCAT2 to become a leading facility dedicated for follow-up transit observations of new interesting planets. The multi-color transit modeling developed in this work is useful for the future studies studies related with MuSCAT2.