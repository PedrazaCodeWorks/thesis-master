\begin{abstract}
%intro
While the Kepler space mission was responsible for most of planet discoveries, the majority of the $\sim$523 planets with masses > 0.5 M$_J$ were discovered by ground-based surveys that are optimized to find giant planets in short-period orbits such as hot Jupiters. 
%motivation
Hot Jupiters remain prime targets for atmospheric studies from the ground thanks to their relatively large scale heights.
%observation
For this study, we used OAO/MuSCAT to conduct simultaneous multi-color observations of several transiting systems with known low-density Hot Jupiters including HAT-P-12b, HAT-P-14b, and WASP-21b. 
%Onitsuka  https://arxiv.org/pdf/1701.01588.pdf
%The data were obtained with the Multicolor Simultaneous Camera for studying Atmospheres of Transiting exoplanets (MuSCAT) on the 188 cm telescope at Okayama Astrophysical Observatory in Japan. We observed the fading event in the $g′_2$-, $r′_2$-, and $z_{s,2}$-bands simultaneously.
%goal
Our goals are (1) to improve the transit parameters of these systems and (2) to search for broad spectral features such as Rayleigh scattering signature in the optical wavelengths. 
%method, analysis
Our homogeneous analysis of transit light curves uses Bayesian modeling to determine the best estimates and Bayesian credible regions of the transit and systematics model parameters, taking into account the presence of correlated noise in the light curves.
%results
%%%-----------------------improved transit ephemerides
%The c 2 value of the linear fit is 15.5 for five degrees of freedom (DOF), meaning that the linear function nominally has a 2.6σ discrepancy with the observed data. However, discrepancies with similar levels often arise in ground-based T c observations, possibly due to unknown systematics rather than the true timing variations. Therefore, we do not consider it to be a noticeable TTV signal at this point. In any case, correction of the ephemeris in this work should be useful for future observations.
%%%-----------------------search for TDV
%Next we search for Transit Duration Variations (TDV) to check for any evidence of additional bodies. The measured b values of the seven transits and their uncertainties are listed in Table 5 and shown in Figure 5. A constant fit to these values gives the mean of b = 0.9015±0.0024 with c 2 = 7.5 for dof = 6, meaning that no significant variation is observed
%%%-----------------------atmospheric model

The derived transit parameters are in general agreement with previous results. 
%As a result, we find a significant wavelength dependence of fading depths of about 3.1\%, 1.7\%, 1.0\% for the $g′_2$-, $r′_2$-, and $z_{s,2}$-bands, respectively. A cloudless H/He dominant atmosphere of a hot Jupiter cannot explain this large wavelength dependence. Additionally, we rule out a scenario by the occultation of the gravity-darkened host star.
%%%-----------------------atmospheric model
Careful analysis of transit depth variation in each band show a marked increase in the planetary radius from the red toward the blue ends of the visible wavelength range.  
In addition, to compare the observed data with a theoretical atmospheric model, we calculate a model spectrum for the first time for each planet considering two cases: (case 1) 1 $\times$ Solar and (case 2) 100 $\times$ Solar metallicity clear atmospheres, assuming thermochemical equilibrium compositions and isothermal structures. 
Comparing the measured transit depths with the spectrum model for HAT-P-12b and HAT-P-44b help rule out stellar activity and site-specific systematics as the source of this variation. In addition, we obtained a broadband transmission spectrum in agreement with the atmospheric model of Sing et al. (2016) %\cite{Sing2016} 
for HAT-P-12b, thus confirming the validity of our transit modeling approach.

To test the significance of (non-)detection of Rayleigh slope, a Monte Carlo fitting routine was performed to compute slopes directly from the marginalized posterior distributions of transit depth for each band. Results show that Rayleigh slope detection is marginal (1.5$\sigma$) for HAT-P-12b and (1.8$\sigma$) HAT-P-44b implying that the achieved photometric precision is not enough to robustly distinguish between the two atmospheric models. %Assuming that detection is true, we provide an estimate of the atmospheric scale height and the atmosphere's mean molecular weight.
However, our modeling is useful to categorically rule out the possibility that the observed trend in WASP-21b spectrum is not atmospheric in origin within 2.8$\sigma$. Instead, the observed trend can be explained by unocculted spots on the surface of WASP-21 with a spot coverage of 0.5\%. 
Motivated by our results, we search for new targets feasible for future observation aimed at detecting broad atmospheric features. The multi-color transit modeling developed in this work is useful for the future studies studies related with MuSCAT2.
%This study is a stepping stone to future observations that will be conducted with MuSCAT
%Our analysis of simultaneous multi-color photometry shows a steep rise in the absorption depth towards shorter wavelengths in the optical which can be interpreted as Rayleigh scattering in its atmosphere. Our spectrum modeling indicates that our transmission spectrum can be best represented by hazy or cloudy atmosphere. HAT-P-44 b is one of the few planets in a multi-planet system whose atmosphere can be characterized from the ground using a 2-m class telescope.
%summary
%Finally, we present feasible targets for MuSCAT/MuSCAT2 to search for Rayleigh scattering. 
\end{abstract}